\documentclass{article}
\usepackage[utf8]{inputenc}
\usepackage[a4paper, total={6.5in, 10in}]{geometry}
\usepackage{enumerate}
\usepackage{amssymb}
\usepackage{amsmath}
\usepackage{xcolor}
\usepackage{parskip}

\title{Logic Session 4 Notes}
\author{Monica}
\date{December 2021}

\begin{document}

\maketitle

\section*{Category Theory}
\begin{itemize}
  \item Essentially everything in maths can be incorporated into categories, therefore making CT arguably \emph{the} approach forward
\end{itemize}

\section*{Q1. Why is propositional logic not sufficient? Can all statements be modelled in propositional logic while retaining the semantics?}
\begin{itemize}
  \item Prop logic does not allow us to talk about individual objects
  \item FOL is able to encode every \underline{sequence} (w/ $\vdash$) into a logical formula, allowing us to conclude A$\Rightarrow$B w/o making any assumptions
  \begin{itemize}
    \item A $\vdash$ B (sequence) $\equiv$ $\vdash$ A $\Rightarrow$ B (logical formula)
    \begingroup
    \renewcommand\labelenumi{(\theenumi)}
    \begin{enumerate}
    \item $\forall n \in \mathbb{N} : n$ is odd $\Rightarrow n^2$ is odd  \\ (can also be written as $n$ is odd $\Rightarrow n^2$ is odd, $n \in \mathbb{N}$) \label{item:1}
    \item $\forall n \in \mathbb{N} : n$ is odd $\Rightarrow \forall n \in \mathbb{N} : n^2$ is odd  \label{item:2}
    \end{enumerate}
    \endgroup
    \item In the session, we investigated the following example, where (\ref{item:1}) is a proposition, and (\ref{item:2}) is our attempt at breaking down the proposition by applying the universal quantifier to both propositional variables (as the scope of the universal quantifier encompasses n and \(n^2\)). We find that (\ref{item:2}) does not make sense and is a vacuous statement, thus making this incompatible with propositional logic.
    \item[$\therefore$] At the end of the day, first order logic can still use propositions, though it is not needed.
    \item[$\therefore$] Just because a statement is a proposition, unless it can be further broken down into its constituent propositions and logical connectives, then it is not necessarily compatible with propositional logic.
  \end{itemize}
  \item[$\therefore$] When using FOL, can just rely on propositions w/o need of sequence. On the other hand, having sequences is also a technically way to avoid use of universal quantifiers
  \item[$\therefore$] That said, sometimes there’s no reason to turn sequences into formulas, esp when working w/ other forms of logic like CT, it’s better to use sequences.
\end{itemize}

\subsubsection*{Extension: Quantifiers}
\begin{itemize}
  \item Universal quantifiers are not always necessary to produce meaningful formulas with variables, since they are already encoded in the \underline{context} of the formulae.
  \begin{itemize}
    \item The context captures the free variables (unbounded by quantifiers) in the formula. Therefore, we don't need the universal quantifier in particular to do logic involving variables
    \item Fundamentally, the quantifiers take a formula and turn it into a proposition (each do this differently)
  \end{itemize}
\end{itemize}

\section*{Q2. Is FOL sufficient to model all of maths?}
\begin{itemize}
  \item No since FOL only talks about the variables of the predicates, but not the predicates themselves (leads to 2nd and higher order logic)
  \item[!]Important point: Even though there may be diff systems that are more high level/complex, it doesn't make a system insufficient to describe a model. Just simply diff perspectives
\end{itemize}

\subsubsection*{Extension: How can we then quantify over infinitely many variables?}
\begin{itemize}
  \item Since we can quantify over functions, we can therefore quantify over infinitely many vars
  \item There are theories that can input infinitely many vars, allowing us to make sense of this fact
  \item[$\therefore$] We’ve already been using this notion when working w/ intersections and unions in Amman-Escher
\end{itemize}

\subsubsection*{Extension: Why “first’ order logic?}
\begin{itemize}
  \item 1st: quantifies over \underline{individual elements} from interpretation domain (ie. natural numbers, real numbers)
  \item 2nd: quantities over entire \underline{sets}
  \item 3rd: quantifies over \underline{sets of sets}, etc
  \item[$\therefore$] Therefore “order” refers to the degree of the families of sets. Initially i thought it could related to the definition of order in partial and total order, though i now think they are fundamentally different concepts
  \begin{itemize}
    \item[$\therefore$] I can also see a parallel to Lean in terms of the hierarchy of Types. {\textcolor{red}{Is the significance of having a hierarchical structure in Lean related to higher order logic?}}
  \end{itemize}
\end{itemize}

\subsubsection*{Extension: NAND gates}
\begin{itemize}
  \item Concept in (theoretical ?) comp sci/maths that enables the entirety of prop logic to be encoded using NAND gates only, as opposed to {$\Rightarrow, \Leftrightarrow, \neg, \wedge, \vee$}
  \begin{itemize}
    \item This is potentially useful in terms of computing (replacing AND, OR gates with NAND gates), though problems with complexity and cost benefit arises
  \end{itemize}
\end{itemize}

\subsubsection*{Extension: Computer science}
\begin{itemize}
  \item Turing machine used as a benchmark to determine whether something is theoretically computable in finite amt of time (at least classically)
  \item If something can be made into/used as a Turing machine, then it is computable, ie. John Conway’s Game of Life
  \item Benefits of using this: 
  \begin{itemize}
    \item finding algorithms that solve these computable problems
    \item Classify problems and study the relationships between them
    \item Potentially find different notions of computability and surpassing Turing machine
    \begin{itemize}
      \item Ie. somehow being able to implement infinitely running programs
    \end{itemize}
  \end{itemize}
  \item[$\therefore$] Whenever there is a limitation, explore whether it can be surpassed
\end{itemize}

\section*{Q3. Are predicates strictly necessary for FOL?}
\begin{itemize}
  \item[$\therefore$] No, bc it is fundamentally a relation, thus we only need relations. They are just for convenience
  \item However in lean, it is difficult to work in terms of relations/function (which are just a specific kind of relation - left total and right unique) in lieu of predicates (i think)
\end{itemize}

\section*{Q4. Definition of theory}
\begin{itemize}
  \item[$\rightarrow$] A theory is defined by a set of axioms OR
  \item[$\rightarrow$] A theory is the set of all sequences that are true in the theory
\end{itemize}

\section*{Q5. Completeness of a deduction system and theory}
\begin{itemize}
  \item[$\rightarrow$] A deduction system is complete if every statement in system that is true* has a proof that can be derived from statements of the same system
  \item[$\rightarrow$] A theory is complete if every question/statement pertaining to the theory has proof within statements that are part of the same theory
  \item true in all models*
  \vspace{4mm} 
  \item[!] FOL is complete (Godel’s completeness theorem)
  \item The incompleteness theorem refers to any system/theory that is an extension of the theory of the natural numbers, since Peano axioms are incomplete
  \begin{itemize}
    \item This is not the case for FOL, but for 2nd and higher order logic
  \end{itemize}
\end{itemize}

\section*{Q6. Type theory}
\begin{itemize}
  \item[$\rightarrow$] Involves \underline{type construction}
  \begin{itemize}
    \item Building types from other types and logical connectors.
    \item Ie. Product type, function type, proposition type
  \end{itemize}
  \item Type is analogous to data type in coding. OOP is like an application of type theory
  \vspace{4mm} 
  \item Lean type checks
  \begin{itemize}
    \item Validity in lean means the verification of whether constructed type actually reproduces the same type as the proposition itself has
    \item[$\therefore$] A proof in Lean is nothing but the type of the result itself
  \end{itemize}
  \vspace{4mm} 
  \item Cool thing: Homotopy type theory
  \begin{itemize}
    \item Combines topological and logical ideas together
    \item Proposed as new foundation of logic (in lieu of set theory and even CT)
  \end{itemize}
\end{itemize}

\section*{Q8. Modelling functions in terms of sets}
\begin{itemize}
  \item[$\rightarrow$] Functions can be modelled as ordered pair formed by the domain and co-domain of function, $(x, f(x))$, which is a subset of the Cartesian product of two sets, X, Y, such that there is exactly 1 element of Y for each element in X
\end{itemize}

\section*{Reflection:}
\begin{itemize}
  \item Understanding still superficial
  \item Did not really make use of meta-cognitive thinking
  \item Need to actively cross-link between different concepts learnt (ie. Amman, Escher, logic, diff geo, etc) so to not forget and continue making connections between diff areas
\end{itemize}

\pagebreak
\section*{Model of FOL}
(this model is more general and pragmatic than shown in Lean - makes it easily applicable for diff systems too)
\subsubsection{Formal language}
\begin{itemize}
  \item[$\rightarrow$] \underline{Signature} $\Sigma = {Types} \cup {Function symbols} \cup {Relation symbols}$, where
  \begin{itemize}
    \item functions - $F:A_1 \times ... \times A_n \rightarrow B$
    \item relations - $R \rightarrowtail A_1 \times ... \times A_n$
    \item arity is the number of arguments/variables
  \end{itemize}
  \item[$\rightarrow$] \underline{Variables $\add$ Context} : set of pairs, var = ${(x,A) \vert x is a symbol for var, A is the type of x}$
  \item Relations build formulas and functions build terms
  
  \item[$\rightarrow$] \underline{Terms}
  \begingroup
    \renewcommand\labelenumi{(\theenumi)}
    \begin{enumerate}
      \item Any variable $x:A$ is a term of type $A$ \label{item:1}
      \item if $t_1:A_1, ..., t_n:A_n$ are terms of types $A_1:A_n$ and  $F:A_1 \times ... \times A_n \rightarrow B$ is a function symbol, then $F(t_1, t_2, ..., t_n)$ is a term of type $B$ \label{item:2}
      \item The set $\Sigma$-term of terms is the set satisfying (\ref{item:1}) and (\ref{item:2})
    \end{enumerate}
  \endgroup

  \item[$\rightarrow$] \underline{Formulae-in-context}
  \begin{enumerate}
    \item Technically only for free variables - essentially clarifying what are the types of variables
  \end{enumerate}
  \begingroup
    \renewcommand\labelenumi{(\theenumi)}
    \begin{enumerate}
      \item $\top$ and $\bot$are formulae in the empty context (smallest context possible) 
      \item if $\psi$ and $\varphi$ are formulae-in-context, then the following are also formulae-in-context: \\ {\textcolor{red}{(not entirely sure what is meant by joined context - does it have smth to do w/ interchangeability?)}}
      \begin{enumerate}
        \item $\psi \wedge \varphi$ \qquad $x_1:A_1,...,x_n:A_n,y_1:B_1,...,y_m:B_m,...,y_m:B_m$
        \item $\psi \vee \varphi$ \qquad $x_1:A_1,...,x_n:A_n,y_1:B_1,...,y_m:B_m,...,y_m:B_m$
        \item $\neg \psi$ \quad \qquad $x_1:A_1,...,x_n:A_n$
        \item $\psi \Rightarrow \varphi$ \qquad $x_1:A_1,...,x_n:A_n,y_1:B_1,...,y_m:B_m,...,y_m:B_m$
        \item $\psi \Leftrightarrow \varphi$ \qquad $x_1:A_1,...,x_n:A_n,y_1:B_1,...,y_m:B_m,...,y_m:B_m$
        \item $\exists x_1: \psi$ \quad \quad $x_2:A_2,...,x_n:A_n$
        \item $\forall x_1: \psi$ \qquad $x_2:A_2,...,x_n:A_n$
      \end{enumerate}
      \item If $t_1:A_1, ..., t_n:A_n$ are terms and \\ $R \rightarrowtail A_1 \times ... \times A_n$ is a relation-in-context \\ Then $R([t_1/x_1],[t_2/x_2],...,[t_n/x_n])$ is a formula-in-context ($[t_n/x_n]$ is substitution of terms x)
      \item If $t_1,t_2:A$ \\ Then $t_1=t_2$ is a formulae-in-context
    \end{enumerate}
  \endgroup
\end{itemize}
{\textcolor{red}{(are both terms and formulae-in-context defined through structural induction?)}}


\end{document}
